% Options for packages loaded elsewhere
\PassOptionsToPackage{unicode}{hyperref}
\PassOptionsToPackage{hyphens}{url}
%
\documentclass[
]{article}
\usepackage{amsmath,amssymb}
\usepackage{lmodern}
\usepackage{iftex}
\ifPDFTeX
  \usepackage[T1]{fontenc}
  \usepackage[utf8]{inputenc}
  \usepackage{textcomp} % provide euro and other symbols
\else % if luatex or xetex
  \usepackage{unicode-math}
  \defaultfontfeatures{Scale=MatchLowercase}
  \defaultfontfeatures[\rmfamily]{Ligatures=TeX,Scale=1}
\fi
% Use upquote if available, for straight quotes in verbatim environments
\IfFileExists{upquote.sty}{\usepackage{upquote}}{}
\IfFileExists{microtype.sty}{% use microtype if available
  \usepackage[]{microtype}
  \UseMicrotypeSet[protrusion]{basicmath} % disable protrusion for tt fonts
}{}
\makeatletter
\@ifundefined{KOMAClassName}{% if non-KOMA class
  \IfFileExists{parskip.sty}{%
    \usepackage{parskip}
  }{% else
    \setlength{\parindent}{0pt}
    \setlength{\parskip}{6pt plus 2pt minus 1pt}}
}{% if KOMA class
  \KOMAoptions{parskip=half}}
\makeatother
\usepackage{xcolor}
\usepackage[margin=1in]{geometry}
\usepackage{color}
\usepackage{fancyvrb}
\newcommand{\VerbBar}{|}
\newcommand{\VERB}{\Verb[commandchars=\\\{\}]}
\DefineVerbatimEnvironment{Highlighting}{Verbatim}{commandchars=\\\{\}}
% Add ',fontsize=\small' for more characters per line
\usepackage{framed}
\definecolor{shadecolor}{RGB}{248,248,248}
\newenvironment{Shaded}{\begin{snugshade}}{\end{snugshade}}
\newcommand{\AlertTok}[1]{\textcolor[rgb]{0.94,0.16,0.16}{#1}}
\newcommand{\AnnotationTok}[1]{\textcolor[rgb]{0.56,0.35,0.01}{\textbf{\textit{#1}}}}
\newcommand{\AttributeTok}[1]{\textcolor[rgb]{0.77,0.63,0.00}{#1}}
\newcommand{\BaseNTok}[1]{\textcolor[rgb]{0.00,0.00,0.81}{#1}}
\newcommand{\BuiltInTok}[1]{#1}
\newcommand{\CharTok}[1]{\textcolor[rgb]{0.31,0.60,0.02}{#1}}
\newcommand{\CommentTok}[1]{\textcolor[rgb]{0.56,0.35,0.01}{\textit{#1}}}
\newcommand{\CommentVarTok}[1]{\textcolor[rgb]{0.56,0.35,0.01}{\textbf{\textit{#1}}}}
\newcommand{\ConstantTok}[1]{\textcolor[rgb]{0.00,0.00,0.00}{#1}}
\newcommand{\ControlFlowTok}[1]{\textcolor[rgb]{0.13,0.29,0.53}{\textbf{#1}}}
\newcommand{\DataTypeTok}[1]{\textcolor[rgb]{0.13,0.29,0.53}{#1}}
\newcommand{\DecValTok}[1]{\textcolor[rgb]{0.00,0.00,0.81}{#1}}
\newcommand{\DocumentationTok}[1]{\textcolor[rgb]{0.56,0.35,0.01}{\textbf{\textit{#1}}}}
\newcommand{\ErrorTok}[1]{\textcolor[rgb]{0.64,0.00,0.00}{\textbf{#1}}}
\newcommand{\ExtensionTok}[1]{#1}
\newcommand{\FloatTok}[1]{\textcolor[rgb]{0.00,0.00,0.81}{#1}}
\newcommand{\FunctionTok}[1]{\textcolor[rgb]{0.00,0.00,0.00}{#1}}
\newcommand{\ImportTok}[1]{#1}
\newcommand{\InformationTok}[1]{\textcolor[rgb]{0.56,0.35,0.01}{\textbf{\textit{#1}}}}
\newcommand{\KeywordTok}[1]{\textcolor[rgb]{0.13,0.29,0.53}{\textbf{#1}}}
\newcommand{\NormalTok}[1]{#1}
\newcommand{\OperatorTok}[1]{\textcolor[rgb]{0.81,0.36,0.00}{\textbf{#1}}}
\newcommand{\OtherTok}[1]{\textcolor[rgb]{0.56,0.35,0.01}{#1}}
\newcommand{\PreprocessorTok}[1]{\textcolor[rgb]{0.56,0.35,0.01}{\textit{#1}}}
\newcommand{\RegionMarkerTok}[1]{#1}
\newcommand{\SpecialCharTok}[1]{\textcolor[rgb]{0.00,0.00,0.00}{#1}}
\newcommand{\SpecialStringTok}[1]{\textcolor[rgb]{0.31,0.60,0.02}{#1}}
\newcommand{\StringTok}[1]{\textcolor[rgb]{0.31,0.60,0.02}{#1}}
\newcommand{\VariableTok}[1]{\textcolor[rgb]{0.00,0.00,0.00}{#1}}
\newcommand{\VerbatimStringTok}[1]{\textcolor[rgb]{0.31,0.60,0.02}{#1}}
\newcommand{\WarningTok}[1]{\textcolor[rgb]{0.56,0.35,0.01}{\textbf{\textit{#1}}}}
\usepackage{graphicx}
\makeatletter
\def\maxwidth{\ifdim\Gin@nat@width>\linewidth\linewidth\else\Gin@nat@width\fi}
\def\maxheight{\ifdim\Gin@nat@height>\textheight\textheight\else\Gin@nat@height\fi}
\makeatother
% Scale images if necessary, so that they will not overflow the page
% margins by default, and it is still possible to overwrite the defaults
% using explicit options in \includegraphics[width, height, ...]{}
\setkeys{Gin}{width=\maxwidth,height=\maxheight,keepaspectratio}
% Set default figure placement to htbp
\makeatletter
\def\fps@figure{htbp}
\makeatother
\setlength{\emergencystretch}{3em} % prevent overfull lines
\providecommand{\tightlist}{%
  \setlength{\itemsep}{0pt}\setlength{\parskip}{0pt}}
\setcounter{secnumdepth}{-\maxdimen} % remove section numbering
\ifLuaTeX
  \usepackage{selnolig}  % disable illegal ligatures
\fi
\IfFileExists{bookmark.sty}{\usepackage{bookmark}}{\usepackage{hyperref}}
\IfFileExists{xurl.sty}{\usepackage{xurl}}{} % add URL line breaks if available
\urlstyle{same} % disable monospaced font for URLs
\hypersetup{
  pdftitle={FE515 2022A Assignment 2},
  pdfauthor={Yufu Liao},
  hidelinks,
  pdfcreator={LaTeX via pandoc}}

\title{FE515 2022A Assignment 2}
\author{Yufu Liao}
\date{03/11/2023}

\begin{document}
\maketitle

\hypertarget{question-1-50-points}{%
\section{Question 1: (50 points)}\label{question-1-50-points}}

\hypertarget{section}{%
\subsection{1.1}\label{section}}

Find the attached JPM.csv file. Use as.Date() function to change the
first column to Date object.

\begin{Shaded}
\begin{Highlighting}[]
\NormalTok{jpm }\OtherTok{\textless{}{-}} \FunctionTok{read.csv}\NormalTok{(}\StringTok{"JPM.csv"}\NormalTok{)}
\NormalTok{jpm}\SpecialCharTok{$}\NormalTok{X }\OtherTok{\textless{}{-}} \FunctionTok{as.Date}\NormalTok{(jpm}\SpecialCharTok{$}\NormalTok{X, }\AttributeTok{origin =} \StringTok{"2023/01/01"}\NormalTok{)}
\NormalTok{jpm[}\DecValTok{1}\SpecialCharTok{:}\DecValTok{10}\NormalTok{,]}
\end{Highlighting}
\end{Shaded}

\begin{verbatim}
##             X JPM.Open JPM.High JPM.Low JPM.Close JPM.Volume JPM.Adjusted
## 1  2023-01-02    48.00    48.37   47.59     48.07   14244700     32.52235
## 2  2023-01-03    48.05    48.55   47.75     48.19    9471500     32.60353
## 3  2023-01-04    48.17    48.25   47.63     47.79   10760500     32.33291
## 4  2023-01-05    47.57    48.06   47.32     47.95    8239200     32.44115
## 5  2023-01-06    47.90    48.11   47.36     47.75    9276700     32.30586
## 6  2023-01-07    47.47    48.12   47.44     48.10   15597000     32.54265
## 7  2023-01-08    48.00    48.42   47.94     48.31    8049200     32.68473
## 8  2023-01-09    48.10    48.26   47.90     47.99   10646700     32.46823
## 9  2023-01-10    48.16    48.46   48.10     48.39    8696500     32.73885
## 10 2023-01-11    48.65    48.89   48.12     48.43   16291400     32.76591
\end{verbatim}

\hypertarget{section-1}{%
\subsection{1.2}\label{section-1}}

Plot the adjusted close price against the date object (i.e.~date object
as x-axis and close price as y-axis) in red line (require no points).
Set the title as JPM, the label for x-axis as Date and the label for
y-axis as Adjusted Close Price.

\begin{Shaded}
\begin{Highlighting}[]
\FunctionTok{plot}\NormalTok{(jpm}\SpecialCharTok{$}\NormalTok{X, jpm}\SpecialCharTok{$}\NormalTok{JPM.Adjusted,}
     \AttributeTok{main =} \StringTok{"JPM"}\NormalTok{,}
     \AttributeTok{xlab =} \StringTok{"Date"}\NormalTok{,}
     \AttributeTok{type =} \StringTok{"l"}\NormalTok{,}
     \AttributeTok{ylab =} \StringTok{"Adjusted Close Price"}\NormalTok{,}
     \AttributeTok{col =} \StringTok{"red"}
\NormalTok{)}
\end{Highlighting}
\end{Shaded}

\includegraphics{YufuLiao_hw2_files/figure-latex/unnamed-chunk-2-1.pdf}

\hypertarget{section-2}{%
\subsection{1.3}\label{section-2}}

Create a scatter plot of close price against open price (i.e.~open
prices as x-axis, and close prices as y-axis). Set the x label as ''Open
Price'' and y label as ''Close Price''.

\begin{Shaded}
\begin{Highlighting}[]
\FunctionTok{plot}\NormalTok{(jpm}\SpecialCharTok{$}\NormalTok{JPM.Open, jpm}\SpecialCharTok{$}\NormalTok{JPM.Close,}
     \AttributeTok{xlab =} \StringTok{"Open Price"}\NormalTok{,}
     \AttributeTok{ylab =} \StringTok{"Close Price"}
\NormalTok{)}
\end{Highlighting}
\end{Shaded}

\includegraphics{YufuLiao_hw2_files/figure-latex/unnamed-chunk-3-1.pdf}

\hypertarget{section-3}{%
\subsection{1.4}\label{section-3}}

Use cut() function to divide adjusted close price into 4 intervals.
Generate a barplot for the frequencies of these intervals.

\begin{Shaded}
\begin{Highlighting}[]
\NormalTok{cut\_close }\OtherTok{\textless{}{-}} \FunctionTok{cut}\NormalTok{(jpm}\SpecialCharTok{$}\NormalTok{JPM.Adjusted, }\DecValTok{4}\NormalTok{)}
\FunctionTok{barplot}\NormalTok{(}\FunctionTok{table}\NormalTok{(cut\_close), }\AttributeTok{xlab =} \StringTok{"Adjusted Close Price"}\NormalTok{, }\AttributeTok{ylab =} \StringTok{"Frequency"}\NormalTok{)}
\end{Highlighting}
\end{Shaded}

\includegraphics{YufuLiao_hw2_files/figure-latex/unnamed-chunk-4-1.pdf}

\hypertarget{section-4}{%
\subsection{1.5}\label{section-4}}

Generate a boxplot of volume against the 4 intervals of adjusted close
prices.

\begin{Shaded}
\begin{Highlighting}[]
\FunctionTok{boxplot}\NormalTok{(}\FunctionTok{table}\NormalTok{(cut\_close), }\AttributeTok{ylab =} \StringTok{"Frequency"}\NormalTok{)}
\end{Highlighting}
\end{Shaded}

\includegraphics{YufuLiao_hw2_files/figure-latex/unnamed-chunk-5-1.pdf}

\hypertarget{section-5}{%
\subsection{1.6}\label{section-5}}

Use par() function to create a picture of 4 subplots. Gather the 4
figures from 1.2 - 1.5 into ONE single picture. Please arrange the 4
subplots into a 2 by 2 frame, i.e.~a frame consists of 2 columns and 2
rows. (Hint. \texttt{par(mfrow\ =\ c(1,3))} will create a picture of
three subplots. In the picture, the subplots are arranged into a 1 by 3
frame.)

\begin{Shaded}
\begin{Highlighting}[]
\FunctionTok{par}\NormalTok{(}\AttributeTok{mfrow =} \FunctionTok{c}\NormalTok{(}\DecValTok{2}\NormalTok{, }\DecValTok{2}\NormalTok{))}
\FunctionTok{plot}\NormalTok{(jpm}\SpecialCharTok{$}\NormalTok{X, jpm}\SpecialCharTok{$}\NormalTok{JPM.Adjusted,}
     \AttributeTok{main =} \StringTok{"JPM"}\NormalTok{,}
     \AttributeTok{xlab =} \StringTok{"Date"}\NormalTok{,}
     \AttributeTok{type =} \StringTok{"l"}\NormalTok{,}
     \AttributeTok{ylab =} \StringTok{"Adjusted Close Price"}\NormalTok{,}
     \AttributeTok{col =} \StringTok{"red"}
\NormalTok{)}
\FunctionTok{plot}\NormalTok{(jpm}\SpecialCharTok{$}\NormalTok{JPM.Open, jpm}\SpecialCharTok{$}\NormalTok{JPM.Close,}
     \AttributeTok{xlab =} \StringTok{"Open Price"}\NormalTok{,}
     \AttributeTok{ylab =} \StringTok{"Close Price"}
\NormalTok{)}
\FunctionTok{barplot}\NormalTok{(}\FunctionTok{table}\NormalTok{(cut\_close), }\AttributeTok{xlab =} \StringTok{"Adjusted Close Price"}\NormalTok{, }\AttributeTok{ylab =} \StringTok{"Frequency"}\NormalTok{)}
\FunctionTok{boxplot}\NormalTok{(}\FunctionTok{table}\NormalTok{(cut\_close), }\AttributeTok{ylab =} \StringTok{"Frequency"}\NormalTok{)}
\end{Highlighting}
\end{Shaded}

\includegraphics{YufuLiao_hw2_files/figure-latex/unnamed-chunk-6-1.pdf}

\hypertarget{question-2}{%
\section{Question 2}\label{question-2}}

Estimate the volume of the unit sphere (which is just 4π/3) by
simulation.

\begin{Shaded}
\begin{Highlighting}[]
\NormalTok{seed }\OtherTok{\textless{}{-}} \DecValTok{1}
\NormalTok{rnd }\OtherTok{\textless{}{-}} \ControlFlowTok{function}\NormalTok{(n)\{}
\NormalTok{  m }\OtherTok{\textless{}{-}} \DecValTok{2} \SpecialCharTok{\^{}} \DecValTok{31} \SpecialCharTok{{-}} \DecValTok{1}
\NormalTok{  a }\OtherTok{\textless{}{-}} \DecValTok{7} \SpecialCharTok{\^{}} \DecValTok{5}
\NormalTok{  b }\OtherTok{\textless{}{-}} \DecValTok{0}
  
\NormalTok{  x }\OtherTok{\textless{}{-}} \FunctionTok{rep}\NormalTok{(}\ConstantTok{NA}\NormalTok{, n)}
\NormalTok{  x[}\DecValTok{1}\NormalTok{] }\OtherTok{\textless{}{-}}\NormalTok{ (a }\SpecialCharTok{*}\NormalTok{ seed }\SpecialCharTok{+}\NormalTok{ b) }\SpecialCharTok{\%\%}\NormalTok{ m}
  
  \ControlFlowTok{for}\NormalTok{(i }\ControlFlowTok{in} \DecValTok{1}\SpecialCharTok{:}\NormalTok{(n }\SpecialCharTok{{-}} \DecValTok{1}\NormalTok{))\{}
\NormalTok{    x[i }\SpecialCharTok{+} \DecValTok{1}\NormalTok{] }\OtherTok{\textless{}{-}}\NormalTok{ (a }\SpecialCharTok{*}\NormalTok{ x[i] }\SpecialCharTok{+}\NormalTok{ b) }\SpecialCharTok{\%\%}\NormalTok{ m}
\NormalTok{  \}}
\NormalTok{  seed }\OtherTok{\textless{}\textless{}{-}}\NormalTok{ x[n]}
  
  \FunctionTok{return}\NormalTok{(x }\SpecialCharTok{/}\NormalTok{ m)}
\NormalTok{\}}

\NormalTok{num.total }\OtherTok{\textless{}{-}} \DecValTok{100000}
\NormalTok{x }\OtherTok{\textless{}{-}} \FunctionTok{rnd}\NormalTok{(num.total)}
\NormalTok{y }\OtherTok{\textless{}{-}} \FunctionTok{rnd}\NormalTok{(num.total)}
\NormalTok{z }\OtherTok{\textless{}{-}} \FunctionTok{rnd}\NormalTok{(num.total)}

\NormalTok{num.inner }\OtherTok{\textless{}{-}} \FunctionTok{sum}\NormalTok{(x }\SpecialCharTok{\^{}} \DecValTok{2} \SpecialCharTok{+}\NormalTok{ y }\SpecialCharTok{\^{}} \DecValTok{2} \SpecialCharTok{+}\NormalTok{ z }\SpecialCharTok{\^{}} \DecValTok{2} \SpecialCharTok{\textless{}=} \DecValTok{1}\NormalTok{)}
\NormalTok{volume.eighth }\OtherTok{\textless{}{-}}\NormalTok{ num.inner }\SpecialCharTok{/}\NormalTok{ num.total}
\NormalTok{(volume.sphere }\OtherTok{\textless{}{-}} \DecValTok{8} \SpecialCharTok{*}\NormalTok{ volume.eighth)}
\end{Highlighting}
\end{Shaded}

\begin{verbatim}
## [1] 4.16512
\end{verbatim}

\hypertarget{question-3}{%
\section{Question 3}\label{question-3}}

\hypertarget{section-6}{%
\subsection{3.1}\label{section-6}}

Implement a Linear Congruential Generator (LCG)

\begin{Shaded}
\begin{Highlighting}[]
\NormalTok{LCG }\OtherTok{\textless{}{-}} \ControlFlowTok{function}\NormalTok{(n) \{}
\NormalTok{  m }\OtherTok{\textless{}{-}} \DecValTok{244944}
\NormalTok{  a }\OtherTok{\textless{}{-}} \DecValTok{1597}
\NormalTok{  b }\OtherTok{\textless{}{-}} \DecValTok{51749}
  
\NormalTok{  x }\OtherTok{\textless{}{-}} \FunctionTok{rep}\NormalTok{(}\ConstantTok{NaN}\NormalTok{, n)}
\NormalTok{  x[}\DecValTok{1}\NormalTok{] }\OtherTok{\textless{}{-}}\NormalTok{ (a }\SpecialCharTok{*} \DecValTok{1} \SpecialCharTok{+}\NormalTok{ b) }\SpecialCharTok{\%\%}\NormalTok{ m}
  
  \ControlFlowTok{for}\NormalTok{(i }\ControlFlowTok{in} \DecValTok{1}\SpecialCharTok{:}\NormalTok{(n }\SpecialCharTok{{-}} \DecValTok{1}\NormalTok{)) \{}
\NormalTok{    x[i }\SpecialCharTok{+} \DecValTok{1}\NormalTok{] }\OtherTok{\textless{}{-}}\NormalTok{ (a }\SpecialCharTok{*}\NormalTok{ x[i] }\SpecialCharTok{+}\NormalTok{ b) }\SpecialCharTok{\%\%}\NormalTok{ m}
\NormalTok{  \}}
  
  \FunctionTok{return}\NormalTok{(x }\SpecialCharTok{/}\NormalTok{ m)}
\NormalTok{\}}
\end{Highlighting}
\end{Shaded}

\hypertarget{section-7}{%
\subsection{3.2}\label{section-7}}

Use the LCG in the previous problem, generate 10000 random numbers from
chi-square distribution with 10 degrees of freedom
(i.e.~\texttt{df\ =\ 10}), and assign to a variable. (Hint.:
\texttt{X\ =\ qnorm(LCG(10000))} will generate a sample of 10000 numbers
X which follows normal distribution. For chi-square case, please
consider another function \texttt{qchisq()}.)

\begin{Shaded}
\begin{Highlighting}[]
\NormalTok{X }\OtherTok{\textless{}{-}} \FunctionTok{qchisq}\NormalTok{(}\FunctionTok{LCG}\NormalTok{(}\DecValTok{10000}\NormalTok{), }\AttributeTok{df =} \DecValTok{10}\NormalTok{)}
\FunctionTok{head}\NormalTok{(X)}
\end{Highlighting}
\end{Shaded}

\begin{verbatim}
## [1]  6.382024  3.042432  9.342521 13.756028 11.820942  9.921533
\end{verbatim}

\hypertarget{section-8}{%
\subsection{3.3}\label{section-8}}

Visualize the resulting sample from 3.2 using a histogram with 40 bins.

\begin{Shaded}
\begin{Highlighting}[]
\FunctionTok{hist}\NormalTok{(X, }\AttributeTok{nclass =} \DecValTok{40}\NormalTok{)}
\end{Highlighting}
\end{Shaded}

\includegraphics{YufuLiao_hw2_files/figure-latex/unnamed-chunk-10-1.pdf}

\end{document}
